

\documentclass[letterpaper]{article}

\usepackage{xcolor}
\usepackage{amsmath}

% Comment the following lines to use the default Computer Modern font
% instead of the Palatino font provided by the mathpazo package.
% Remove the 'osf' bit if you don't like the old style figures.
\usepackage[T1]{fontenc}
%\usepackage[sc]{mathpazo}
%\usepackage[sc,osf]{mathpazo}

\usepackage{titlesec}

%%% to do month year
\usepackage[en-US]{datetime2}




\usepackage[margin=1.0in]{geometry}

\titleformat{\section}{\large}{\thesection}{1em}{\hrule}
%\titlespacing*{\section}{0pt}{.1cm }{.0cm}

% Set your name here


\def\name{Wesley Howden}


% Replace this with a link to your CV if you like, or set it empty
% (as in \def\footerlink{}) to remove the link in the footer:
\def\footerlink{}

% The following metadata will show up in the PDF properties
\usepackage{hyperref}
\hypersetup{
  colorlinks = false,
  urlcolor = black,
  linkbordercolor= {blue},
  pdfauthor = {\name},
  pdfkeywords = {economics, statistics, mathematics},
  pdftitle = {\name: Curriculum Vitae},
  pdfsubject = {Curriculum Vitae},
  pdfpagemode = UseNone
}

\geometry{
  %body={6.5in, 8.5in},
  left=1.0in,
  top=1.0in
}

% Customize page headers
\pagestyle{myheadings}
\markright{\name}
\thispagestyle{empty}

% Custom section fonts
\usepackage{sectsty}
\sectionfont{\rmfamily\mdseries\Large}
\subsectionfont{\rmfamily\mdseries\itshape\large}

\usepackage{tabto}



% Other possible font commands include:
% \ttfamily for teletype,
% \sffamily for sans serif,
% \bfseries for bold,
% \scshape for small caps,
% \normalsize, \large, \Large, \LARGE sizes.

% Don't indent paragraphs.
\setlength\parindent{0em}

% Make lists without bullets
\renewenvironment{itemize}{
  \begin{list}{}{
    \setlength{\leftmargin}{1em}
      \setlength{\itemsep}{0.25em}
    \setlength{\parskip}{0pt}
    \setlength{\parsep}{0pt} 
  }
}{
  \end{list}
}


\begin{document}



% Place name at left
{

\centering

{\huge \bf { \name}}



}

{

\centering 
\textsc{
university of arizona, department of economics\\

\vspace{.3cm}

%\hrule
}

% Alternatively, print name centered and bold:
%\centerline{\huge \bf \name}



%\vspace{0.25in}

\section*{\textsc{\textbf{contact information}}}
\hrule
\vspace{.2cm}
\begin{minipage}{0.65\textwidth}
  {University of Arizona} \\
  Department of Economics \\
  1130 E Helen St \\
  McClelland Hall 401-HH
  Tucson, AZ 85721
\end{minipage}
\begin{minipage}{0.35\textwidth}
%    Phone: & (985) 373 5653 \\
%    Fax: &  (919) 962 5678 \\
    Email:  \hfill \href{mailto:whowden@ucsd.edu}{{whowden@arizona.edu}} \\
    Website:  \hfill \href{https://wesleyhowden.com}{{{wesleyhowden.com}}} \\ 
    Office phone: \hfill \href{tel:15206212529}{+1 520 621 2529}
\end{minipage}

\section*{\textsc{\textbf{academic appointments}}}
\hrule
\vspace{0.2cm}
Postdoctoral Research Associate \hfill 2021--present \\
\emph{University of Arizona} \hfill Tucson, AZ 


\section*{\textsc{\textbf{education}}}
\hrule
\vspace{0.2cm}
%\subsection*{University of California, San Diego}
Ph.D. in Economics \hfill 2021  \\
\emph{University of California San Diego} \hfill La Jolla, CA \\

%\begin{itemize}
%\item Committee: \\ Mark Jacobsen (chair), Judson Boomhower, Richard Carson, Josh Graff Zivin, Jennifer Burney 
%\end{itemize}
%
%M.A. Economics \hfill 2017 \\ 
%\emph{University of California San Diego} \hfill La Jolla, CA \\

S.B. in Mathematics, A.B. in Economics, A.B. in Political Science (with honors) \hfill 2015 \\
 \emph{The University of Chicago} \hfill Chicago, IL

%\section*{\textsc{\textbf{References}}}
%\hrule
%\vspace{0.2cm}
%\begin{itemize}
%\item  Mark Jacobsen, University of California San Diego \hfill \href{mailto:m3jacobsen@ucsd.edu}{\textcolor{blue}{m3jacobsen@ucsd.edu}} 
%\item  Judson Boomhower, University of California San Diego \hfill  \href{mailto:jboomhower@ucsd.edu}{\textcolor{blue}{jboomhower@ucsd.edu}}  
%\item  Richard Carson, University of California San Diego \hfill\href{mailto:rcarson@ucsd.edu}{\textcolor{blue}{rcarson@ucsd.edu}}
%\end{itemize}

\section*{\textsc{\textbf{fields of interest}}}
\hrule
\vspace{.25cm}
\begin{itemize}
\item Environment and Energy, Development
\end{itemize}


\section*{\textsc{\textbf{working papers}}}
\hrule
\vspace{.3cm}
\begin{itemize}

\item  \href{https://papers.ssrn.com/sol3/papers.cfm?abstract_id=3892429}{``Adaptation to Weather Shocks and Household Beliefs on Climate: Evidence from California''}
\vspace{0.1cm}

%Using a difference-in-difference framework, I show that California households exposed to a severe heat wave are differentially more likely to adopt central air conditioning units than those less exposed, controlling for historical climate. Using these ``induced adopters'' to predict take-up, I show that induced adopters have a significant increase in their summer energy demand three years following the heat wave, with insignificant effects on their winter electricity demand. In addition, I present a theoretical framework where household belief-updating about the climate rationalizes household learning about the climate that cannot be explained by myopia or alternative channels.
\vspace{0.2cm}


\item \href{https://papers.ssrn.com/sol3/papers.cfm?abstract_id=4132517}{``The Global Impacts of Climate Change on Risk Preferences''} (with Remy Levin) 
\vspace{0.1cm}

%How do individual risk preferences adapt to climate change? To study this question, we start by building a model of risk preference adaptation. In our model, a Bayesian agent is exposed to unavoidable, exogenous background risk with an unknown mean and unknown variance, and learns about its moments from personal experience. As the agent's beliefs about the background risk evolve, their preferences over endogenous risks adapt in turn. We test the predictions of our model in two large, longitudinal surveys from Indonesia and Mexico, by linking within-person, long-run changes in elicited risk preferences to data on individual lifetime experiences of climate change. In line with our model's predictions, we find that (1) increases in the experienced mean of heat and precipitation in both settings induce significant decreases in measured risk aversion; (2) increases in the variance of heat in Indonesia and the variance of precipitation in Mexico lead to significant increases in measured risk aversion; (3) the magnitude of the variance effect is 0.7-1.6 times that of the mean effect, indicating that perceptions of uncertainty are of first-order importance; and (4) increases in measured risk aversion correlate with decreases in risk-taking behavior in the domains of migration and smoking. Building on recent advances in welfare economics, in the final part of the paper we develop a new method for estimating whether observed risk preference changes are, in fact, adaptive. Using our method we find that in our sample this is the case, with climate-change-induced risk preference changes increasing collective welfare by approximately one percentage point. 
\vspace{0.2cm}
    


\item \href{https://papers.ssrn.com/sol3/papers.cfm?abstract_id=4142761}{``Inundated by Change: The Effects of Land Use on Flood Damages"}
\vspace{0.1cm}
%Proper land-use policy can help mitigate damages from natural disasters. Management of such policy occurs at different geographic levels, with potential implications for optimal management given the level of aggregation. This study examines this by quantifying the effects of land-use change on flood damages in the state of Texas. I link claims data from the National Flood Insurance Program to a series of land-use changes to construct a tract-by-month panel, and use exogenous variation in precipitation across tract-months to estimate the effect of changes in land use on the frequency and magnitude of new flood insurance claims. I find that increases in impervious surface development within a tract increase flood insurance claims, while increases in water coverage and other natural covers decrease these claims. In addition, using variation in tract-level elevation, I show that land-use change in neighboring geographies---particularly those uphill---affects own-tract flood insurance claims. Overall, these results suggest existence of spatial spillovers from land-use changes within a geography, and imply returns to coordination in land-use policy across geographies.
\vspace{0.2cm}

``Impacts of Maternal Exposure on PFAS on Infant Health Outcomes'' (with Robert Baluja, Bo Guo, Ashley Langer, and Derek Lemoine)


\end{itemize}

\section*{\textsc{\textbf{works in progress}}}
\hrule 
\vspace{.2cm}
\begin{itemize}
\item ``A Historical Measure of Risk Preferences'' (with Remy Levin \& Daniela Vidart)
\vspace{0.3cm}
\item ``Measuring error in weather data'' (with Derek Lemoine and Wint Myat Thu)
\vspace{0.3cm}
\item ``Jet stream linkages and global production'' (with Abbie Boatwright, Ellie Broadman, and Derek Lemoine)
\vspace{0.3cm}
\item ``Directed Technical Change in Energy and Choice of Market Policy''
\vspace{0.3cm}
\item ``International Spillover in Secondary Car Market Regulation: Evidence from a German Scrappage Policy''
\end{itemize}

\section*{\textsc{\textbf{publications}}}
\hrule
\vspace{.2cm}
Pre-doctoral publications
\begin{itemize}
\item \href{https://www.dropbox.com/s/mhxkbqmrir1vqwr/Leguizamon_etal_2020.pdf?dl=0}{``Revisiting the Link Between Economic Distress, Race and Domestic Violence''} \, \hfill 2020  \\ \phantom{\,,}with Sebastian Leguizamon and Susane Leguizamon, \emph{Journal of Interpersonal Violence}
\end{itemize}

%\section*{\textsc{\textbf{relevant positions}}}
%\hrule
%\vspace{.2cm}
%\begin{itemize}
%\item Research Assistant to Judson Boomhower \hfill  2019--2020
%\item Research Assistant to Dale Squires \hfill 2018
%\end{itemize}
%
%

\section*{\textsc{\textbf{teaching}}}
\hrule 
\vspace{.2cm}


%\begin{tabbing}
%Microeconomics for Policy/Management (Masters TA): Fall 2020 \\
%Environmental Economics (Masters TA): Fall 2018\\
%Energy Economics (TA): Winter 2019 \\
%Intermediate Macroeconomics (TA): 2016, 2017 ($\times 2$), 2018, 2019 ($\times 2$), 2020 ($\times 2$) \\
%Fall 2016, Winter 2017, Spring 2017, Summer 2018, Spring 2019, Fall 2019, Winter 2020, Spring 2020\\
%Intermediate Microeconomics (TA): Fall 2018
%\end{tabbing}


University of California San Diego \hfill La Jolla, CA\\ 
\emph{Instructor of Record} \hfill 2020 

\begin{itemize}
    \item Principles of Macroeconomics -- \href{https://www.dropbox.com/s/5pvyrr303b5ceyc/Howden_Wesley_CAPE_-_ECON_3_-_Principles_of_Macroeconomics_%5BA01%5D_%28Howden_Wesley_Dixon%29_-_S120A.pdf?dl=0}{{100\% positive student reviews}} (20/27 response rate)
\end{itemize}

\emph{Teaching Assistant}  \hfill 2016--2021

\begin{itemize}
    \item Microeconomics for Policy \& Management (Masters level), Environmental Economics (Masters level), Energy Economics, Intermediate Macroeconomics, Intermediate Microeconomics
\end{itemize}

\emph{Certifications} 
\begin{itemize}
    \item UC San Diego Learning Certificate: Introduction to College Teaching \hfill{2019}
\end{itemize}




\section*{\textsc{\textbf{presentations}}}
\hrule 
\vspace{.2cm}
\begin{tabbing}
2022\quad \=Arizona State University CEESP* \\ \> AERE Summer Conference\\ \> University of Arizona AREC  \\ \ \>Sustainability and Development Conference, University of Michigan \\
2021\quad \= University of Arizona \\
\> Online Summer Workshop in Environment, Energy, and Transportation Economics (OSWEET)  \\
2020\quad \= Workshop in Sustainable Development, Columbia University  \\
\> UCSD Environmental Seminar (Spring, Fall) \\
2019\quad \= UCSD Environmental Seminar (Spring, Fall) \\ 
2018  \> UCSD Environmental Seminar (Spring, Fall) \\
 \> UCSD Graduate Student Research Seminar \\
 \> UCSD Interdisciplinary Forum for Environmental Research \\
 \> Berkeley Summer School in Environmental and Energy Economics  
\end{tabbing}

* \emph{scheduled}

\section*{\textsc{\textbf{invited participation}}}
\hrule 
\vspace{.2cm}
\begin{tabbing}
2022\quad \= NBER Environmental and Energy Policy and the Economy Conference \\ 
2021\quad \= NBER Environmental and Energy Policy and the Economy Conference \\
\> NBER Workshop on the Economics of Energy Innovation \\
2020\quad \= NBER Environmental and Energy Policy and the Economy Conference\\ 
2016\quad \= Berkeley Summer School in Environmental and Energy Economics

\end{tabbing}

\section*{\textsc{\textbf{fellowships and awards}}}
\hrule
\vspace{.2cm}
\begin{tabbing}
2020\quad \= UC San Diego Travel Grant (canceled, COVID-19) \\
2019\quad \> Benjamin C. Horne Memorial Prize (for work in energy and environment), UC San Diego\\
  \> Research Fellowship, UC San Diego \\
2017  \> San Diego Fellowship, Program for Interdisciplinary Environmental Research \\
 \> Graduate Student Research Fellowship, UC San Diego\\
2016 \> Center for Environmental Economics Travel Grant UC San Diego \\ 
\> Graduate Student Research Fellowship, UC San Diego\\
2015 \> Regents Fellowship, UC San Diego \\
\> Phi Beta Kappa, The University of Chicago\\
2011 \> University Scholarship, The University of Chicago
\end{tabbing}


\section*{\textsc{\textbf{grants}}}
\hrule
\vspace{.2cm}
\begin{tabbing}
2019\quad \= IAH Interdisciplinary Group Award, UC San Diego Institute for the Arts \& Humanities \\ % (\$1000) \\
2018  \> IAH Interdisciplinary Group Award, UC San Diego Institute for the Arts \& Humanities %(\$1500)
\end{tabbing}


\section*{\textsc{\textbf{service}}}
\hrule
\vspace{.2cm}

\begin{itemize}
\item AERE Session Chair, Adaptation and Resilience: Avoidance Behavior (2022)
\item  \href{https://environment.arizona.edu/air}{Arizona Institute for Resilient Environments and Societies} (AIRES) Environmental Economics Workshop Organizer (2021--2022)
\item AIRES coordinator for economics (2021-- )
\item UC San Diego Interdisciplinary Forum for Environmental Research (organizer, 2017--2018)
\end{itemize}

\section*{\textsc{\textbf{additional information}}}
\hrule
\vspace{.2cm}


\begin{itemize}
    \item Languages: English (native), French (conversational)
    \item Citizenship: United States
    \item Programming: Stata, R, \LaTeX, Matlab, Python, Git, ArcGIS (basic)
\end{itemize}



%\section*{Publications}






\bigskip

% Footer
\begin{center}
  \begin{footnotesize}
    Last updated: \DTMlangsetup{showdayofmonth=false}
\today
\DTMlangsetup{showdayofmonth=true} \\
   % \href{\footerlink}{\texttt{\footerlink}}
  \end{footnotesize}
\end{center}

\end{document}